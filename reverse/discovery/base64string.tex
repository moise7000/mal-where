%! Author = ewan
%! Date = 12/5/25

\documentclass[11pt,a4paper]{article}

% Packages de base
\usepackage[utf8]{inputenc}
\usepackage[T1]{fontenc}
\usepackage[french]{babel}
\usepackage[margin=2.5cm]{geometry}
\usepackage{hyperref}
\usepackage{fancyhdr}
\usepackage{malware-template}

% Configuration en-tête/pied de page
\pagestyle{fancy}
\fancyhf{}
\lhead{Analyse Reverse Engineering}
\rhead{\thepage}
\lfoot{Ewan Decima}
\rfoot{\today}
\setlength{\headheight}{14pt}

\begin{document}





% ========== DÉCOUVERTE 1 ==========
    \discoverystart
    {Très longue chaîne encodée}
    {0x}
    {const}
    {Encodage}
    {Faible}

    \subsubsection*{Code Assembleur}
    \begin{asmcode}
        ; Check for debugger
mov eax, fs:[0x30]
movzx eax, byte ptr [eax+2]
test al, al
jnz short debugger_detected
    \end{asmcode}



    \discoveryanalysis{%
        Cette longue chaine de caractère semble être une chaine encodée en \texttt{base64}. Après avoir mis cette chaine 
        dans CyberChef celui-ci ne parvenenait pas à décoderla chaine. Nous avons alors construit un petit script Python.
        Après une première itération, la chaine semblait toujours être encodé en \texttt{base64}. Nous avons alors
        modifié le script pour y ajouter une analyse d’entropie. Après une trentaine d'itération nous sommes parvenu à décoder la
        chaine : \textbf{Hop là, on n'oublie pas de s'amuser}

    }

    \discoveryscreenshot{screenshots/decouverte_01.png}

    %\discoveryiocs
    %\item \textbf{Registry:} N/A
    %\item \textbf{File:} N/A
    %\item \textbf{Network:} N/A
    %\item \textbf{String:} \texttt{"BeingDebugged"}
    %\discoveryend







\end{document}