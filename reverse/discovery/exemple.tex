\documentclass[11pt,a4paper]{article}

% Packages de base
\usepackage[utf8]{inputenc}
\usepackage[T1]{fontenc}
\usepackage[french]{babel}
\usepackage[margin=2.5cm]{geometry}
\usepackage{hyperref}
\usepackage{fancyhdr}

% Importer le template malware
\usepackage{malware-template}

% Configuration en-tête/pied de page
\pagestyle{fancy}
\fancyhf{}
\lhead{Analyse Reverse Engineering}
\rhead{\thepage}
\lfoot{Ewan Decima}
\rfoot{\today}
\setlength{\headheight}{14pt}

\begin{document}





% ========== DÉCOUVERTE 1 ==========
    \discoverystart
    {Anti-Debug via IsDebuggerPresent}
    {0x00401234}
    {check_environment}
    {Anti-debug}
    {High}

    \subsubsection*{Code Assembleur}
    \begin{asmcode}
        ; Check for debugger
mov eax, fs:[0x30]
movzx eax, byte ptr [eax+2]
test al, al
jnz short debugger_detected
    \end{asmcode}

    \subsubsection*{Code Décompilé}
    \begin{ccode}
        if (IsDebuggerPresent()) {
    exit_process();
}
    \end{ccode}

    \discoveryanalysis{%
        Cette fonction vérifie la présence d'un débogueur en lisant directement le PEB (Process Environment Block). Si le flag \texttt{BeingDebugged} est activé, le malware termine immédiatement son exécution pour éviter l'analyse dynamique.
    }

    \discoveryscreenshot{screenshots/decouverte_01.png}

    %\discoveryiocs
    %\item \textbf{Registry:} N/A
    %\item \textbf{File:} N/A
    %\item \textbf{Network:} N/A
    %\item \textbf{String:} \texttt{"BeingDebugged"}
    %\discoveryend







\end{document}