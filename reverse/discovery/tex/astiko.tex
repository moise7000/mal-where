\documentclass[11pt,a4paper]{article}

% Packages de base
\usepackage[utf8]{inputenc}
\usepackage[T1]{fontenc}
\usepackage[french]{babel}
\usepackage[margin=2.5cm]{geometry}
\usepackage{hyperref}
\usepackage{fancyhdr}

% Importer le template malware
\usepackage{malware-template}

% Configuration en-tête/pied de page
\pagestyle{fancy}
\fancyhf{}
\lhead{Analyse Reverse Engineering}
\rhead{\thepage}
\lfoot{Ewan Decima}
\rfoot{\today}
\setlength{\headheight}{14pt}

\begin{document}





% ========== DÉCOUVERTE 1 ==========
    \discoverystart
    {Ecriture dans un fichier}
    {0x00401D1C}
    {?}
    {Ecriture}
    {Faible}

    \subsubsection*{Code Assembleur}
    \begin{asmcode}
mov     [esp+2Ch+Msg.pt.x], offset Mode ; "w"
mov     [esp+2Ch+Msg.time], offset Filename ; "C:\\Users\\lhs\\AppData\\Local\\Temp\\a"...
call    fopen
test    eax, eax
mov     [ebp+var_B8], eax
jz      loc_401F2C
mov     edi, [ebp+var_B8]
mov     [esp+2Ch+Msg.pt.y], 13h ; Count
mov     [esp+2Ch+Msg.pt.x], 1 ; Size
mov     [esp+2Ch+Msg.time], offset aPakboEtLombrik ; "pakbo-et-lombrik.fr"
mov     [esp+2Ch+File], edi ; File
mov     [ebp+var_98], 6
call    fwrite
mov     [esp+2Ch+Msg.time], edi ; File
call    fclose
    \end{asmcode}


    \discoveryanalysis{%
        Ce code ouvre un fichier dans le dossier temporaire \texttt{lhs\textbackslash{}AppData\textbackslash{}Local\textbackslash{}Temp\textbackslash{}astiko.txt}
        de l’utilisateur et y écrit la chaîne "pakbo-et-lombrik.fr". Si le dossier n'existe pas celui-ci est créé avant
        l'écriture.
    }

    \discoveryscreenshot{../screenshots/astiko.png}









\end{document}