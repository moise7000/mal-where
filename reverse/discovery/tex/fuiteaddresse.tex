%! Author = ewan
%! Date = 12/5/25

\documentclass[11pt,a4paper]{article}

% Packages de base
\usepackage[utf8]{inputenc}
\usepackage[T1]{fontenc}
\usepackage[margin=2.5cm]{geometry}
\usepackage{hyperref}
\usepackage{fancyhdr}
\usepackage{malware-template}

% Configuration en-tête/pied de page
\pagestyle{fancy}
\fancyhf{}
\lhead{Analyse Reverse Engineering}
\rhead{\thepage}
\lfoot{Ewan Decima}
\rfoot{\today}
\setlength{\headheight}{14pt}

\begin{document}





% ========== DÉCOUVERTE 1 ==========
    \discoverystart
    {Affichage furtif et fuite d'adresse mémoire}
    {0x004052BC, 0x0040530C}
    {?}
    {?}
    {Moyenne}





    \discoveryscreenshot{../screenshots/cmdLine.png}



    \discoveryanalysis{%
        En éxecutant le ficher \texttt{main.exe} de manière légitime (en respectant les contraintes de taille, avec la
        chaîne \texttt{a} par exemple), le programme affiche pendant mopoins d'une seconde la chaîne de caractère suivante :
        \texttt{a   Félicitations ! tu as trouve la cle04114E0, 004114F8, 00411470, 00411468, 004114F0, 00411460}. 
    En cherchant dans le résultat de \texttt{strings main.exe} on trouve la chaîne partielle \texttt{licitations !} et la chaîne
    \texttt{\%p, \%p, \%p, \%p, \%p, \%p}, ansi en cherchant dans les données brutes avec IDA on peut remonter au code
    assembleur suivant :


    }

    \subsubsection{Code Assembleur : Fuite d'adresse mémoire}
    \begin{asmcode}
mov     [esp+518h+hWnd], eax
mov     [esp+518h+lpRect], edx
call    __moddi3
mov     [ebp+var_54], eax
mov     [esp+518h+x1], offset strlen
mov     [esp+518h+hdcSrc], offset free
mov     [esp+518h+cy], offset malloc
mov     [esp+518h+wDest], offset memcpy
mov     [esp+518h+y], offset loc_4114F8
mov     [esp+518h+lpRect], offset printf
mov     [esp+518h+hWnd], offset aPPPPPP ; "%p, %p, %p, %p, %p, %p"
mov     [ebp+var_44C], 4
call    printf
    \end{asmcode}


    \discoveryanalysis{%
        Ce code appel la fonction \texttt{printf} et affiche les addresses mémoires des fonctions suivantes :
    \begin{itemize}
        \item \texttt{strlen}
        \item \texttt{free}
        \item \texttt{malloc}
        \item \texttt{memcpy}
        \item \texttt{loc\_4114F8}
        \item \texttt{printf}
    \end{itemize}


    }














\end{document}