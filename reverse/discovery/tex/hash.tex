%! Author = ewan
%! Date = 12/5/25

\documentclass[11pt,a4paper]{article}

% Packages de base
\usepackage[utf8]{inputenc}
\usepackage[T1]{fontenc}g
\usepackage[margin=2.5cm]{geometry}
\usepackage{hyperref}
\usepackage{fancyhdr}
\usepackage{malware-template}

% Configuration en-tête/pied de page
\pagestyle{fancy}
\fancyhf{}
\lhead{Analyse Reverse Engineering}
\rhead{\thepage}
\lfoot{Ewan Decima}
\rfoot{\today}
\setlength{\headheight}{14pt}

\begin{document}





% ========== DÉCOUVERTE 1 ==========
    \discoverystart
    {Test SHA-256}
    {0040331B}
    {?}
    {Hash}
    {Moyenne}


\begin{asmcode}
.text:00401F6F ; __unwind { // sub_474F60
.text:00401F6F                 lea     eax, [ebp+var_A]
.text:00401F72                 mov     [esp+8], eax    ; int
.text:00401F76                 mov     dword ptr [esp+4], offset aHelloWorld ; "Hello, World!"
.text:00401F7E                 mov     dword ptr [esp], offset dword_565030 ; int
.text:00401F85                 mov     [ebp+fctx.call_site], 2
.text:00401F8C                 call    sub_448120
.text:00401F91                 mov     dword ptr [esp], offset sub_4020A0 ; _onexit_t
.text:00401F98                 call    sub_409C20
.text:00401F9D                 lea     eax, [ebp+var_A+1]
.text:00401FA0                 mov     [esp+8], eax    ; int
.text:00401FA4                 mov     dword ptr [esp+4], offset aBea8e217036cb3 ; "bea8e217036cb3b738e207fe5d40266828bc196"...
.text:00401FAC                 mov     dword ptr [esp], offset dword_565034 ; int
.text:00401FB3                 mov     [ebp+fctx.call_site], 1
.text:00401FBA                 call    sub_448120
.text:00401FBF                 mov     dword ptr [esp], offset sub_401FF0 ; _onexit_t
.text:00401FC6                 call    sub_409C20
.text:00401FCB                 lea     eax, [ebp+fctx]
.text:00401FCE                 mov     [esp], eax      ; lpfctx
.text:00401FD1                 call    _Unwind_SjLj_Unregister
.text:00401FD6                 leave
.text:00401FD7                 retn
\end{asmcode}


    \discoveryanalysis{%
Dans la fonction ci-dessus, le programme stocke la valeur
    \begin{center}
        bea8e217036cb3b738e207fe5d40266828bc1969fd8538d533ea39f4e40ffc8f
    \end{center}
    qui semble être un digest (SHA-256) en \texttt{dword\_565034}.
    }
    
\begin{asmcode} 
.text:00403770 loc_403770:
.text:00403770                 cmp     ecx, ecx
.text:00403772                 repe cmpsb
.text:00403774                 jnz     loc_403333
.text:0040377A                 mov     eax, [ebp+var_24]
.text:0040377D                 mov     edi, [eax-0Ch]
.text:00403780                 lea     edx, [eax-0Ch]
.text:00403783                 test    edi, edi
.text:00403785                 jnz     loc_403996
\end{asmcode}

    \discoveryanalysis{
    Ensuite, le programme compare le SHA-256 de l'entrée utilisateur avec la valeur de hash. Si les deux sont identiques, 
    le programme saute à la fonction situé en \texttt{loc\_403996}
    }

    \begin{asmcode}
loc_403996:
.text:00403996                 mov     esi, [edx+8]
.text:00403999                 test    esi, esi
.text:0040399B                 js      short loc_4039B5
.text:0040399D                 lea     eax, [ebp+var_24]
.text:004039A0                 mov     [esp], eax
.text:004039A3                 mov     [ebp+fctx.call_site], 25h ; '%'
.text:004039AD                 call    sub_445EE0
.text:004039B2                 mov     eax, [ebp+var_24]
.text:004039B5
.text:004039B5 loc_4039B5:
.text:004039B5                 movzx   eax, byte ptr [eax]
.text:004039B8                 mov     dword ptr [esp], offset Format ; "Byte 0: %02x\n"
.text:004039BF                 mov     [ebp+fctx.call_site], 25h ; '%'
.text:004039C9                 mov     [esp+4], eax
.text:004039CD                 call    printf
.text:004039D2                 jmp     loc_40378B
    \end{asmcode}

    \discoveryanalysis{
    La fonction \texttt{loc\_4039B5} permet d'afficher le premier octet de l'entrée.
    }


    \discoveryscreenshot{../screenshots/hash.png}

\discoveryanalysis{
Après avoir identifé ce potentiel comportement, nous avons tenté de le produire. Pour cela nous avons cherché le hash
directement dans l'executable afin de le modifier afin de le remplacer par un hash de notre choix : le SHA-256 de 1234

\begin{center}
    03ac674216f3e15c761ee1a5e255f067953623c8b388b4459e13f978d7c846f4
\end{center}


}

    \discoveryscreenshot{../screenshots/hash_1.png}
    \discoveryscreenshot{../screenshots/hash_2.png}


    \discoveryanalysis{
    Cependant, en effectuant \texttt{m.exe 1234} le comportement reste inchangé. Deux hypothèses sont donc à considérer :
    \begin{itemize}
        \item le programme effectue un test d'intégrité sur son contenu ;
        \item le hash identifé n'impacte en aucun cas la fonction légitime \texttt{echo}.
    \end{itemize}
    }





\end{document}